\documentclass{article}
\usepackage{listings}
\usepackage{xcolor}

% Define colors for code highlighting
\definecolor{background}{RGB}{245,245,245}
\definecolor{codegray}{gray}{0.5}
\definecolor{codepurple}{rgb}{0.58,0,0.82}

% Define code listing style
\lstdefinestyle{mystyle}{
    backgroundcolor=\color{background},
    commentstyle=\color{codegray},
    keywordstyle=\color{blue},
    numberstyle=\tiny\color{codegray},
    stringstyle=\color{codepurple},
    basicstyle=\ttfamily\footnotesize,
    breakatwhitespace=false,
    breaklines=true,
    captionpos=b,
    keepspaces=true,
    numbers=left,
    numbersep=5pt,
    showspaces=false,
    showstringspaces=false,
    showtabs=false,
    tabsize=2
}
\lstset{style=mystyle}

\title{How to Set Up a GitHub Repository with SSH}
\author{}
\date{}

\begin{document}

\maketitle

\section*{1. Set Up SSH}
\begin{enumerate}

	\item Check for an existing SSH key
        Before creating a new key, check if one already exists:
        \begin{lstlisting}[language=bash]
       ls -al ~/.ssh
        \end{lstlisting}
	Look for files like id_rsa and id_rsa.pub. If they exist, you already have a key pair.
        \end{lstlisting}


    \item Generate an SSH key if you don’t already have one:  
    \begin{lstlisting}[language=bash]
    ssh-keygen -t ed25519 -C "ryossj@gmail.com"
    \end{lstlisting}
    Follow the prompts to save the key to the default location and add a passphrase if desired.

    \item Start the SSH agent:
    \begin{lstlisting}[language=bash]
    eval $(ssh-agent -s)
    \end{lstlisting}

    \item Add your SSH key to the agent:
    \begin{lstlisting}[language=bash]
    ssh-add ~/.ssh/id_ed25519
    \end{lstlisting}

    \item Copy your SSH public key to the clipboard:
    \begin{lstlisting}[language=bash]
    cat ~/.ssh/id_ed25519.pub
    \end{lstlisting}
    Copy the output and add it to your GitHub account under **Settings > SSH and GPG keys**.

    \item Test the connection:
    \begin{lstlisting}[language=bash]
    ssh -T git@github.com
    \end{lstlisting}
    You should see a message confirming successful authentication.
\end{enumerate}

\section*{2. Set Up the Repository Locally}
\begin{enumerate}
    \item Navigate to the directory where you want to create the repository:
    \begin{lstlisting}[language=bash]
    cd ~/projects  # Or any directory where you organize projects
    mkdir HP-Nix
    cd HP-Nix
    \end{lstlisting}

    \item Initialize the local repository:
    \begin{lstlisting}[language=bash]
    git init
    \end{lstlisting}

    \item Add a README or other initial files:
    \begin{lstlisting}[language=bash]
    echo "# HP-Nix" > README.md
    git add README.md
    git commit -m "Initial commit"
    \end{lstlisting}
\end{enumerate}

\section*{3. Link to GitHub with SSH}
\begin{enumerate}
    \item Add the SSH remote URL to your repository:
    \begin{lstlisting}[language=bash]
    git remote add origin git@github.com:Rouzihiro/HP-Nix.git
    \end{lstlisting}

    \item Verify the remote:
    \begin{lstlisting}[language=bash]
    git remote -v
    \end{lstlisting}
    It should display:
    \begin{lstlisting}[language=bash]
    origin  git@github.com:Rouzihiro/HP-Nix.git (fetch)
    origin  git@github.com:Rouzihiro/HP-Nix.git (push)
    \end{lstlisting}
\end{enumerate}

\section*{4. Push to GitHub}
\begin{enumerate}
    \item Push your initial commit to GitHub:
    \begin{lstlisting}[language=bash]
    git branch -M main  # Ensure you're on the main branch
    git push -u origin main
    \end{lstlisting}
\end{enumerate}

\end{document}

